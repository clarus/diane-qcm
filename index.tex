\documentclass[a4paper]{article}
\usepackage[utf8]{inputenc}
\usepackage[T1]{fontenc}
\usepackage{lmodern}
\usepackage[francais]{babel}
\usepackage{fullpage}
\usepackage[hidelinks]{hyperref}
\usepackage{graphicx}
\usepackage{amsmath,amsfonts,amssymb}
\usepackage{tikz}
\usepackage{eurosym}

\begin{document}
  \title{Évaluations Mathématiques Bac Pro}
  \author{Module~: dérivées}
  \date{
    Nom~: .......................................\\
    \vspace{0.2cm}
    Date~: .......................................}
  \maketitle
  
  \paragraph{A)}
  Pour chaque question une seule réponse est exacte. Indiquer laquelle.
  \begin{enumerate}
    \item
      Si $\displaystyle f(x) = - \frac{1}{2} x + \frac{2}{3}$~:
      \begin{center}
        \begin{tabular}{p{3cm}p{3cm}p{3cm}}
          a) $\displaystyle f'(x) = \frac{2}{3}$ & b) $\displaystyle f'(x) = \frac{1}{2}$ & c) $\displaystyle f'(x) = - \frac{1}{2}$
        \end{tabular}
      \end{center}

    \item
      Si $f(x) = 2 x^3$~:
      \begin{center}
        \begin{tabular}{p{3cm}p{3cm}p{3cm}}
          a) $f'(x) = x$ & b) $f'(x) = 6 x^2$ & c) $f'(x) = 6x$
        \end{tabular}
      \end{center}

    \item
      Si $\displaystyle f(x) = - \frac{1}{x}$~:
      \begin{center}
        \begin{tabular}{p{3cm}p{3cm}p{3cm}}
          a) $\displaystyle f'(x) = \frac{1}{x^2}$ & b) $\displaystyle f'(x) = \frac{2}{x}$ & c) $\displaystyle f'(x) = - \frac{1}{x^2}$
        \end{tabular}
      \end{center}
  \end{enumerate}

  \paragraph{B)}
  Calculer la dérivée $f'(x)$ de la fonction $f$.
  \begin{enumerate}
    \item $f(x) = - 3 x$
    \item $f(x) = 2x + 5$
    \item $\displaystyle f(x) = \frac{2}{x}$
    \item $f(x) = 5 x^2 - 2x$
    \item $f(x) = 3 x^2 + 7x - 2$
    \item $\displaystyle f(x) = 4 - \frac{1}{x}$
    \item $f(x) = 2 \sqrt{x}$
  \end{enumerate}

  \paragraph{C)}
  Pour les exercices a) et b) suivants, on considère une fonction $f$ définie sur un intervalle $I$. Pour chaque fonction donnée:
  \begin{enumerate}
    \item calculer la dérivée $f'(x)$
    \item étudier le signe de $f'(x)$ sur $I$
    \item dresser le tableau de variations de $f$ et préciser les éventuels extremum (maximum ou minimum)
  \end{enumerate}
  \vspace{0.5cm}
  \[
    \begin{array}{llll}
      \mathrm{a)} & f(x) = x^2 - 9 & \mathrm{sur}\ I = [-4; 4]\\[0.5cm]
      \mathrm{b)} & f(x) = 2x^2 - 10x + 3 & \mathrm{sur}\ I = [0; 4]
    \end{array}
  \]
\end{document}
